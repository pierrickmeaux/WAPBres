The conjecture of $t$-correcting WAPB functions states that for any $n$ such that $wt(n) = t$, there exists a vector $v \in \{0, \pm 1\}^{n+1}$ where $v_k = 0$ when $\binom{n}{k}$ is even, such that
$$
v \cdot \KrM{n}{n} = (v_0, v_1, \ldots, v_n)
\begin{bmatrix}
    \\
    \kraw{k}{\ell}{n}\\
    \\
\end{bmatrix} = (\underset{\operatorname{0}}{0}, \underset{\operatorname{1}}{0}, \ldots, \underset{\operatorname{t - 1}}{0}, X, X, \ldots, X).
$$

Let
\begin{equation*}
\begin{split}
    \mathcal{L} &= \{0 \leq \ell \leq n | \binom{n}{\ell} \text{ is odd}\}\\
    &=\{0 \leq k \leq n | \text{ supp}(\ell) \subseteq \text{supp}(\ell)\}.
\end{split}
\end{equation*}

It follows that $|\mathcal{L}| = 2^t$. Hence the conjecture is equivalent to

\begin{equation}\label{eq:statement}
    \text{for } i \in [0, t-1], \sumK v_k \cdot \kraw{k}{\ell}{n} = 0.
\end{equation}

We want to prove an equivalence to the conjecture as formulated in \cref{prop:equivalence} below.
\begin{proposition}\label{prop:equivalence}
    The following two statements are equivalent:
    \begin{itemize}
        \item For any positive integer $n$ and for $k \in [0, t-1], \sumK v_\ell \cdot \kraw{k}{\ell}{n} = 0$.
        \item For $k \in [0,t-1], \sumK \ell^kv_\ell= 0$.
    \end{itemize}
\end{proposition}

\begin{proof}
    We will conduct this proof in two halves, showing both statements are equivalent to a common third statement.

    \textbf{First part.}

    Let us begin by proving that
    \begin{equation*}
        \begin{split}
            \text{for } k \in [0, t-1], \quad &\sumK v_\ell \cdot \kraw{k}{\ell}{n} = 0\\
            \Longleftrightarrow\\
            \text{for } k \in [0, t-1] \text{ and for } m \in [0, k], \quad &\sumK \ell^mv_\ell \cdot \kraw{k-m}{\ell}{n} = 0,
        \end{split}
    \end{equation*}

    To prove the left to right implication, we will proceed by induction.

    \underline{Initialization.}

    Let $m = 0$. $\sumK \ell^0v_\ell \cdot \kraw{k-0}{\ell}{n} = \sumK v_\ell \cdot \kraw{k}{\ell}{n} = 0$ from the starting condition.

    Let $m = 1$. We know from $m - 1 = 0$ that $\sumK v_\ell \cdot \kraw{k}{\ell}{n} = 0$. It follows that
    $$
    0 = \sumK v_\ell \cdot \kraw{k}{\ell}{n} = k\sumK v_\ell \cdot \kraw{k}{\ell}{n} = \sumK v_\ell \cdot k\kraw{k}{\ell}{n}.
    $$
    From Proposition 4, statement 2 of \cite{DCC:DalMaiSar06}, we know that
    \begin{equation}\label{eq:fact1}
        (k+1)\kraw{k+1}{\ell}{n} = (n-2\ell)\kraw{k}{\ell}{n} - (n-k+1)\kraw{k-1}{\ell}{n}, \quad k \in [1, n-1].
    \end{equation}
    We make use of \cref{eq:fact1} to obtain
    \begin{equation*}
    \begin{split}
        0 &= \sumK v_\ell \cdot \Big( (n - 2\ell)\kraw{k-1}{\ell}{n} - (n - k)\kraw{k-2}{\ell}{n} \Big)\\
        & = n \eqzero{\sumK \Big( v_\ell \cdot \kraw{k-1}{\ell}{n} \Big)} - 2\sumK \Big( \ell v_\ell \cdot \kraw{k-1}{\ell}{n} \Big)\\& - (n-k)\eqzero{\sumK \Big( v_\ell \cdot \kraw{k-2}{\ell}{n} \Big)}\\
        &= -2\sumK \ell v_\ell \cdot \kraw{k-1}{\ell}{n} = \sumK \ell^1v_\ell \cdot \kraw{k-1}{\ell}{n} = 0,
    \end{split}
    \end{equation*}
    for $k \in [2, t-1]$. For the same values of $k$ we can build the following $t-2$ equations:

    \begin{equation*}
        \begin{cases}
            \sumK \ell v_\ell \cdot \kraw{1}{\ell}{n} = 0\\
            \sumK \ell v_\ell \cdot \kraw{2}{\ell}{n} = 0\\
            \ldots\\
            \sumK \ell v_\ell \cdot \kraw{t-2}{\ell}{n} = 0.\\
        \end{cases}
    \end{equation*}
    The next step is unnecessary to the formality of the proof but we estimate it emphasizes the key relations used in the induction step and might help the reader.

    Let $m = 2$. We know from $m - 1 = 1$ that $\sumK \ell v_\ell \cdot \kraw{k-1}{\ell}{n} = 0$. It follows that
    $$
    0 = (k-1)\sumK \ell v_\ell \cdot \kraw{k-1}{\ell}{n} = \sumK \ell v_\ell \cdot (k-1)\kraw{k-1}{\ell}{n} = 0.
    $$
    Using \cref{eq:fact1} we obtain
    \begin{equation*}
    \begin{split}
        0 &= \sumK \ell v_\ell \cdot \Big( (n - 2\ell)\kraw{k-2}{\ell}{n} - (n - k - 1)\kraw{k-3}{\ell}{n} \Big)\\
        & = n \sumK \Big( \ell v_\ell \cdot \kraw{k-2}{\ell}{n} \Big) - 2\sumK \Big( \ell^2v_\ell \cdot \kraw{k-2}{\ell}{n} \Big)\\& - (n - k - 1)\sumK \Big( \ell v_\ell \cdot \kraw{k-3}{\ell}{n} \Big),
    \end{split}
    \end{equation*}
    for $k \in [3, t - 1]$.

    We showed that for $m - 1 = 1$, $\sumK \ell v_\ell \cdot \kraw{k-2}{\ell}{n} = 0$ for $k \in [3, t-1]$.
    And $\sumK \ell v_\ell \cdot \kraw{k-3}{\ell}{n} = 0$ for $k \in [4, t-1]$.
    For $k = 3$, $\sumK \ell v_\ell \cdot \kraw{0}{\ell}{n} = \sumK \ell v_\ell$.
    Since $0 = \sumK v_\ell \cdot \kraw{1}{\ell}{n} = \sumK v_\ell \cdot(n - 2\ell) = n\eqzero{\sumK v_\ell} -2\sumK \ell v_\ell$, we can conclude that $\sumK\ell v_\ell = 0$.

    Thus it follows that
    \begin{equation*}
        \begin{split}
            0 &= n \eqzero{\sumK \Big( \ell v_\ell \cdot \kraw{k-2}{\ell}{n} \Big)} - 2\sumK \Big( \ell^2v_\ell \cdot \kraw{k-2}{\ell}{n} \Big)\\& - (n - k - 1)\eqzero{\sumK \Big( \ell v_\ell \cdot \kraw{k-3}{\ell}{n} \Big)}\\
            &= -2\sumK \Big( \ell^2v_\ell \cdot \kraw{k-2}{\ell}{n} \Big) = \sumK \Big( \ell^2v_\ell \cdot \kraw{k-2}{\ell}{n} \Big) = 0,
        \end{split}
    \end{equation*}
    for $k \in [3, t-1]$. For the same values of $k$ we can build the following $t-3$ equations:

    \begin{equation*}
        \begin{cases}
            \sumK \ell v_\ell \cdot \kraw{1}{\ell}{n} = 0\\
            \sumK \ell v_\ell \cdot \kraw{2}{\ell}{n} = 0\\
            \ldots\\
            \sumK \ell v_\ell \cdot \kraw{t-3}{\ell}{n} = 0\\
        \end{cases}
    \end{equation*}
    The base step is done.

    \underline{Induction.} We want to show that if the initial statement is true for $m$, then it is also true for $m+1$. From the starting assumption we have

    $$
    0 = \sumK \ell^mv_\ell \cdot \kraw{k-m}{\ell}{n} = (k-m)\sumK \ell^mv_\ell \cdot \kraw{k-m}{\ell}{n} = \sumK \ell^mv_\ell \cdot (k-m)\kraw{k-m}{\ell}{n}.
    $$

    Using \cref{eq:fact1}, we develop this further to

    \begin{equation*}
    \begin{split}
        0 &= \sumK \ell^mv_\ell \cdot \Big( (n - 2\ell)\kraw{k-(m+1)}{\ell}{n} - (n - k - m)\kraw{k-(m+2)}{\ell}{n} \Big)\\
        & = n \sumK \Big( \ell^mv_\ell \cdot \kraw{k-(m+1)}{\ell}{n} \Big) - 2\sumK \Big( \ell^{m+1}v_\ell \cdot \kraw{k-(m+1)}{\ell}{n} \Big)\\& - (n - k - m)\sumK \Big( \ell^mv_\ell \cdot \kraw{k-(m+2)}{\ell}{n} \Big),
    \end{split}
    \end{equation*}
    for $k \in [m+2, t-1]$.

    We know that from the step $m-1$, $\sumK \ell^mv_\ell \cdot \kraw{k-(m+1)}{\ell}{n} = 0$.
    and that for $k \in [m+3, t-1]$, $\sumK \ell^mv_\ell \cdot \kraw{k-(m+2)}{\ell}{n} = 0$.
    For $k = m + 2$, we end up with $\sumK \ell^mv_\ell \cdot \kraw{0}{\ell}{n} = \sumK \ell^mv_\ell$. Since $0 = \sumK \ell^{m-1}v_\ell \cdot \kraw{1}{\ell}{n} = \sumK \ell^{m-1}v_\ell \cdot (n - 2\ell)$,
    we develop one more time to $n\eqzero{\sumK \ell^{m-1}v_\ell} - 2\sumK \ell^mv_\ell$.
    Indeed, $\sumK \ell^mv_\ell = 0$ and $\sumK \Big( \ell^mv_\ell \cdot \kraw{k-(m+2)}{\ell}{n} \Big) = 0$.
    Thus it follows that
    \begin{equation*}
    \begin{split}
        & = n \eqzero{\sumK \Big( \ell^mv_\ell \cdot \kraw{k-(m+1)}{\ell}{n} \Big)} - 2\sumK \Big( \ell^{m+1}v_\ell \cdot \kraw{k-(m+1)}{\ell}{n} \Big)\\& - (n - k - m)\eqzero{\sumK \Big( \ell^mv_\ell \cdot \kraw{k-(m+2)}{\ell}{n} \Big)}\\
        &= -2\sumK \ell^{m+1}v_\ell \cdot \kraw{k_(m+1)}{\ell}{n} = \sumK \ell^{m+1}v_\ell \cdot \kraw{k-(m+1)}{\ell}{n} = 0
    \end{split}
    \end{equation*}
    This concludes the proof that
    \begin{equation*}
        \begin{split}
            \text{for } k \in [0, t-1], \quad &\sumK v_\ell \cdot \kraw{k}{\ell}{n} = 0\\
            \implies\\
            \text{for } k \in [0, t-1] \text{ and for } m \in [0, k], \quad &\sumK \ell^mv_\ell \cdot \kraw{k-m}{\ell}{n} = 0,
        \end{split}
    \end{equation*}
    The right to left implication that $\text{for } k \in [0, t-1] \text{ and for } m \in [0, k], \quad \sumK \ell^mv_\ell \cdot \kraw{k-m}{\ell}{n} = 0 \implies \text{for } k \in [0, t-1], \quad \sumK v_\ell \cdot \kraw{k}{\ell}{n} = 0$ is a trivial instance where $m = 0$.
    As such, we can conclude that
    \begin{equation*}
        \begin{split}
            \text{for } k \in [0, t-1], \quad &\sumK v_\ell \cdot \kraw{k}{\ell}{n} = 0\\
            \Longleftrightarrow\\
            \text{for } k \in [0, t-1] \text{ and for } m \in [0, k], \quad &\sumK \ell^mv_\ell \cdot \kraw{k-m}{\ell}{n} = 0,
        \end{split}
    \end{equation*}

    \textbf{Second part.}

    Let us now focus on the second half of the proof to prove
    \begin{equation*}
        \begin{split}
            \text{for } k \in [0, t-1], \quad &\sumK \ell^kv_\ell = 0\\
            \Longleftrightarrow\\
            \text{for } k \in [0, t-1] \text{ and for } m \in [0, k], \quad &\sumK \ell^mv_\ell \cdot \kraw{k-m}{\ell}{n} = 0,
        \end{split}
    \end{equation*}

    \underline{Initialization.}

    Let $m = k$, then $\sumK \ell^kv_\ell \cdot \kraw{k-k}{\ell}{n} = \sumK \ell^kv_\ell \cdot \kraw{0}{\ell}{n} = \sumK \ell^kv_\ell = 0$.

    Let $m = k - 1$, then
    \begin{equation*}
        \begin{split}
            \sumK \ell^{k-1}v_\ell \cdot \kraw{k-(k-1)}{\ell}{n} &= \sumK \ell^{k-1}v_\ell \cdot \kraw{1}{\ell}{n}\\
            &= \sumK \ell^kv_\ell \cdot (n - 2\ell)\\
            &= n\sumK \Big( \ell^{k-1}v_\ell \Big) - 2\eqzero{\sumK \Big( \ell^kv_\ell \Big)}.
        \end{split}
    \end{equation*}
    Finally we have $\sumK \ell^{k-1}v_\ell = 0$, for all $k \in [1, t-1]$.

    The next step is unnecessary to the formality of the proof but we estimate it emphasizes the key relations used in the induction step and might help the reader.

    Let $m = k - 2$, then $\sumK \ell^{k-2}v_\ell \cdot \kraw{k-(k-2)}{\ell}{n} = \sumK \ell^{k-2}v_\ell \cdot \kraw{2}{\ell}{n}$.
    We apply \cref{eq:fact1} to further develop this to
    \begin{equation*}
        \begin{split}
            \sumK \ell^{k-2}v_\ell \cdot \kraw{2}{\ell}{n} &= \sumK \Big( \ell^{k-2}v_\ell \Big( \frac{1}{2}(n - 2\ell)\kraw{1}{\ell}{n} - \frac{1}{2}n\kraw{0}{\ell}{n} \Big)\Big)\\
            &= \frac{n}{2}\sumK \Big( \ell^{k-2}v_\ell \cdot \kraw{1}{\ell}{n} \Big) - \sumK \Big( \ell^{k-1}v_\ell \cdot \kraw{1}{\ell}{n} \Big) - \frac{n}{2}\sumK \Big( \ell^{k-2}v_\ell \Big),
        \end{split}
    \end{equation*}
    for $k \in [2, t-1]$.
    From the previous step $m + 1 = k - 1$ we had seen that $\sumK \ell^{k-1}v_\ell \cdot \kraw{1}{\ell}{n} = 0\text{ for } k \in [1, t-1]$. From the step before where $m + 2 = k$, we also know that $\sumK \Big( \ell^{k-2}v_\ell \Big) = 0$.
    As such, we determine that some quantities nullify and we can rewrite the last expression of our development to

    \begin{equation*}
    \begin{split}
        &\sumK \ell^{k-2}v_\ell \cdot \kraw{2}{\ell}{n}\\
        &= \frac{n}{2}\eqzero{\sumK \Big( \ell^{k-2}v_\ell \cdot \kraw{1}{\ell}{n} \Big)} - \eqzero{\sumK \Big( \ell^{k-1}v_\ell \cdot \kraw{1}{\ell}{n} \Big)} - \frac{n}{2}\eqzero{\sumK \Big( \ell^{k-2}v_\ell \Big)} = 0.
    \end{split}
    \end{equation*}
    The base step is done.

    \underline{Induction.} We want to prove that if the proposition is true for $m$, then it is also true for $m - 1$. Using \cref{eq:fact1}, we develop the relation to
    \begin{equation*}
        \begin{split}
            & \sumK \ell^{m-1}v_\ell \cdot \kraw{k-(m-1)}{\ell}{n}\\
            &= \sumK \Big( \ell^{m-1}v_\ell \cdot \frac{1}{k-(m-1)}\Big( (n-2\ell)\kraw{k-m}{\ell}{n} - (n-k-m)\kraw{k-(m+1)}{\ell}{n} \Big)\Big)\\
            &= \sumK \Big( \ell^{m-1}v_\ell \cdot \frac{n}{k-(m-1)}\kraw{k-m}{\ell}{n} - \ell^m v_\ell \cdot \frac{2}{k-(m-1)}\kraw{k-m}{\ell}{n}\\
            &\quad -\ell^{m-1}v_\ell \cdot\frac{n-k-m}{k-(m-1)}\kraw{k-(m+1)}{\ell}{n} \Big)\\
            &= \frac{n}{k-(m-1)}\sumK \Big(\ell^{m-1}v_\ell\cdot \kraw{k-m}{\ell}{n} \Big) - \frac{2}{k-(m-1)}\sumK \Big(\ell^mv_\ell\cdot \kraw{k-m}{\ell}{n}\Big)\\
            &\quad -\frac{n-k-m}{k-(m-1)}\sumK\Big(\ell^{m-1}v_\ell\cdot \kraw{k-(m+1)}{\ell}{n}\Big).
        \end{split}
    \end{equation*}
    We know from the previous step $m$ that $\sumK \ell^{m-1}v_\ell\cdot \kraw{k-m}{\ell}{n} = 0$ and from the step before $m+1$ that $\sumK \ell^{m-1}v_\ell\cdot \kraw{k-(m+1)}{\ell}{n} = 0$.
    Using this information and the induction condition, we can determine that some quantities are equal to zero and reformulate the last statement as follows.

    \begin{equation*}
        \begin{split}
            & \sumK \ell^{m-1}v_\ell \cdot \kraw{k-(m-1)}{\ell}{n}\\
            &= \frac{n}{k-(m-1)}\eqzero{\sumK \Big(\ell^{m-1}v_\ell\cdot \kraw{k-m}{\ell}{n} \Big)} - \frac{2}{k-(m-1)}\eqzero{\sumK \Big(\ell^mv_\ell\cdot \kraw{k-m}{\ell}{n}\Big)}\\
            &\quad -\frac{n-k-m}{k-(m-1)}\eqzero{\sumK\Big(\ell^{m-1}v_\ell\cdot \kraw{k-(m+1)}{\ell}{n}\Big)}\\
            &= 0.
        \end{split}
    \end{equation*}
    The induction step is done. This concludes the proof that
    \begin{equation*}
        \begin{split}
            \text{for } k \in [0, t-1], \quad &\sumK \ell^kv_\ell = 0\\
            \implies\\
            \text{ for } k \in [0, t-1] \text{ and for } m \in [0, k], \quad &\sumK \ell^mv_\ell\cdot \kraw{k-m}{\ell}{n} = 0.
        \end{split}
    \end{equation*}
    The right to left implication that $\text{for } k \in [0, t-1] \text{ and for} m \in [0, k], \sumK \ell^mv_\ell \cdot \kraw{k-m}{\ell}{n} = 0 \implies \text{for } k \in [0, t-1], \sumK \ell^kv_\ell = 0$ is a trivial instance where $m = k$.
    As such, we can conclude that

    \begin{equation*}
        \begin{split}
            \text{for } k \in [0, t-1], \quad &\sumK \ell^kv_\ell = 0\\
            \Longleftrightarrow\\
            \text{ for } k \in [0, t-1] \text{ and for } m \in [0, k], \quad &\sumK \ell^mv_\ell\cdot \kraw{k-m}{\ell}{n} = 0.
        \end{split}
    \end{equation*}
    We use both equivalences proved here to conclude that
    \begin{equation*}
        \begin{split}
            \text{for } k \in [0, t-1], \quad &\sumK v_\ell \kraw{k}{\ell}{n} = 0\\
            \Longleftrightarrow\\
            \text{ for } k \in [0, t-1], \quad &\sumK \ell^kv_\ell = 0.
        \end{split}
    \end{equation*}
\end{proof}


As such, we can reformulate the initial condition as
\begin{equation}\label{eq:general}
    \text{for } k \in [0, t-1], \quad \sumK \ell^kv_\ell = 0.
\end{equation}

For $k = 0, \ldots t$, we can build the following system using \cref{eq:general} repeatedly.
\begin{equation}
    \begin{cases}
        \sumK \ell^0v_\ell = 0\\
        \sumK \ell^1v_\ell = 0\\
        \sumK \ell^2v_\ell = 0\\
        \ldots\\
        \sumK \ell^tv_\ell = 0.\\
    \end{cases}
\end{equation}

which reformulates our problem as discussing the existence of a vector $(v_{\ell_1}, \ldots, v_{k\ell{2^t}})$ such that
$$
(v_{\ell_1}, \ldots, v_{\ell_{2^t}})
\begin{bmatrix}
    | & | & | & | & \ldots & |\\
    1 & \ell & \ell^2 & \ell^3 & \ldots & \ell^t\\
    | & | & | & | & \ldots & |\\
\end{bmatrix} = 0,
$$

using a  $2^t \times (t + 1)$ matrix where each column is the vector of all $\ell \in \mathcal{L}$.

Let us take an example for $n = 2^i + 2^j$. We have $t = 2$, the matrix would have the following shape
$$
\begin{bmatrix}
    1 & 0 & 0\\
    1 & 2^i & 2^{2i}\\
    1 & 2^j & 2^{2j}\\
    1 & 2^i + 2^j & (2^i + 2^j)^2\\
\end{bmatrix}
$$
that is a Vandermonde matrix. If the conjecture is true, then $\nexists (v_0, v_{2^i}, v_{2^j}, v_{2^i + 2^j}) \in \{-1, 1\}^4$ that belongs to the left kernel of the vandermonde matrix defined by $\mathcal{L}$.
