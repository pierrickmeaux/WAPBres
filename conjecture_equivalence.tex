The conjecture of $t$-correcting WAPB functions states that for any $n$ such that $wt(n) = t$, there exists a vector $v \in \{0, \pm 1\}^{n+1}$ where $v_k = 0$ when $\binom{n}{k}$ is even, such that
$$
v \cdot \Gamma = (v_0, v_1, \ldots, v_n)
\begin{bmatrix}
    \\
    K_i(k, n) \\
    \\
\end{bmatrix} = (\underset{\operatorname{0}}{0}, \underset{\operatorname{1}}{0}, \ldots, \underset{\operatorname{t - 1}}{0}, X, X, \ldots, X),
$$
where $K_i(k,n)$ is the Krawtchouk polynomial of degree $i$ evaluated in $k$ and where $\Gamma$ is the $(n+1) \times (n + 1)$ Krawtchouk matrix which rows and columns are respectively indexed by $k$ and $i$.

Let
\begin{equation*}
\begin{split}
    \mathcal{K} &= \{0 \leq k \leq n | \binom{n}{k} \text{ is odd}\}\\
    &=\{0 \leq k \leq n | \text{ supp}(k) \subseteq \text{supp}(n)\}.
\end{split}
\end{equation*}

It follows that $|\mathcal{K}| = 2^t$. Hence the conjecture is equivalent to

\begin{equation}\label{eq:statement}
    \sum_{k \in \mathcal{K}} v_k \cdot K_i(k, n) = 0, \quad i = 0, \ldots, t-1.
\end{equation}

We want to prove an equivalence to the conjecture as formulated in \cref{prop:equivalence} below.
\begin{proposition}\label{prop:equivalence}
For any positive integer $n$,
$$\sum_{k \in \mathcal{K}} v_k \cdot K_i(k, n) = 0 \Longleftrightarrow \sum_{k \in \mathcal{K}} k^iv_k= 0,$$
for $i = 0, \ldots, t-1$.
\end{proposition}

\begin{proof}
    Let us prove first that
    $$\sum_{k \in \mathcal{K}} v_k \cdot K_i(k, n) = 0 \implies \sum_{k \in \mathcal{K}} k^mv_k \cdot K_{i-m}(k, n) = 0,$$ for $i = 0, \ldots, t-1$ and for any $m = 0, \ldots, i$. We will proceed by induction.

    \underline{Initialization.}

    Let $m = 0$. $\sum_{k \in \mathcal{K}} k^0v_k \cdot K_{i-0}(k, n) = \sum_{k \in \mathcal{K}} v_k \cdot K_i(k, n) = 0$ from the starting condition.

    Let $m = 1$. We know from $m - 1 = 0$ that $\sum_{k \in \mathcal{K}} v_k \cdot K_i(k, n) = 0$. It follows that
    $$
    0 = \sum_{k \in \mathcal{K}} v_k \cdot K_i(k, n) = i\sum_{k \in \mathcal{K}} v_k \cdot K_i(k, n) = \sum_{k \in \mathcal{K}} v_k \cdot iK_i(k, n).
    $$
    From Proposition 4, statement 2 of \cite{DCC:DalMaiSar06}, we know that
    \begin{equation}\label{eq:fact1}
        (i+1)K_{i+1}(k, n) = (n - 2k)K_i(k,n) - (n - i + 1)K_{i-1}(k,n), \forall i = 1, \ldots, n - 1.
    \end{equation}
    We make use of \cref{eq:fact1} to obtain
    \begin{equation*}
    \begin{split}
        0 &= \sum_{k \in \mathcal{K}} v_k \cdot \Big( (n - 2k)K_{i-1}(k, n) - (n - i)K_{i-2}(k, n) \Big)\\
        & = n \eqzero{\sum_{k \in \mathcal{K}} \Big( v_k \cdot K_{i-1}(k, n) \Big)} - 2\sum_{k \in \mathcal{K}} \Big( kv_k \cdot K_{i-1}(k, n) \Big)\\& - (n-i)\eqzero{\sum_{k \in \mathcal{K}} \Big( v_k \cdot K_{i-2}(k, n) \Big)}\\
        &= -2\sum_{k \in \mathcal{K}} kv_k \cdot K_{i-1}(k, n) = \sum_{k \in \mathcal{K}} k^1v_k \cdot K_{i-1}(k, n) = 0,
    \end{split}
    \end{equation*}
    for $i = 2, \ldots, t-1$. For the same values of $i$ we can build the following $t-2$ equations:

    \begin{equation*}
        \begin{cases}
            \sum_{k \in \mathcal{K}} kv_k \cdot K_1(k, n) = 0\\
            \sum_{k \in \mathcal{K}} kv_k \cdot K_2(k, n) = 0\\
            \ldots\\
            \sum_{k \in \mathcal{K}} kv_k \cdot K_{t-2}(k, n) = 0.\\
        \end{cases}
    \end{equation*}

    Let $m = 2$. We know from $m -1 = 1$ that $\sum_{k \in \mathcal{K}} kv_k \cdot K_{i-1}(k, n) = 0$. It follows that
    $$
    0 = (i-1)\sum_{k \in \mathcal{K}} kv_k \cdot K_{i-1}(k, n) = \sum_{k \in \mathcal{K}} kv_k \cdot (i-1)K_{i-1}(k, n) = 0.
    $$
    Using \cref{eq:fact1} we obtain
    \begin{equation*}
    \begin{split}
        0 &= \sum_{k \in \mathcal{K}} kv_k \cdot \Big( (n - 2k)K_{i-2}(k, n) - (n - i - 1)K_{i-3}(k, n) \Big)\\
        & = n \sum_{k \in \mathcal{K}} \Big( kv_k \cdot K_{i-2}(k, n) \Big) - 2\sum_{k \in \mathcal{K}} \Big( k^2v_k \cdot K_{i-2}(k, n) \Big)\\& - (n - i - 1)\sum_{k \in \mathcal{K}} \Big( kv_k \cdot K_{i-3}(k, n) \Big),
    \end{split}
    \end{equation*}
    for $i = 3, \ldots, t-1$.

    We showed that for $m - 1= 1$, $\sum_{k \in \mathcal{K}} kv_k \cdot K_{i-2}(k,n) = 0$ for $i = 3, \ldots, t-1$. And $\sum_{k \in \mathcal{K}} kv_k \cdot K_{i-3}(k,n) = 0$ for $i = 4, \ldots, t-1$. For $i = 3$, $\sum_{k \in \mathcal{K}} kv_k \cdot K_0(k, n) = \sum_{k \in \mathcal{K}} kv_k$. Since $0 = \sum_{k \in \mathcal{K}} v_k \cdot K_1(k, n) = \sum_{k \in \mathcal{K}} v_k \cdot(n - 2k) = n\underset{=0}{\underline{\sum_{k \in \mathcal{K}}v_k}} -2\sum_{k \in \mathcal{K}}kv_k$, we can conclude that $\sum_{k \in \mathcal{K}}kv_k = 0$.

    Thus it follows that
    \begin{equation*}
        \begin{split}
            0 &= n \eqzero{\sum_{k \in \mathcal{K}} \Big( kv_k \cdot K_{i-2}(k, n) \Big)} - 2\sum_{k \in \mathcal{K}} \Big( k^2v_k \cdot K_{i-2}(k, n) \Big)\\& - (n - i - 1)\eqzero{\sum_{k \in \mathcal{K}} \Big( kv_k \cdot K_{i-3}(k, n) \Big)}\\
            &= - 2\sum_{k \in \mathcal{K}} \Big( k^2v_k \cdot K_{i-2}(k, n) \Big) = \sum_{k \in \mathcal{K}} \Big( k^2v_k \cdot K_{i-2}(k, n) \Big) = 0,
        \end{split}
    \end{equation*}
    for $i = 3, \ldots, t-1$. For the same values of $i$ we can build the following $t-3$ equations:

    \begin{equation*}
        \begin{cases}
            \sum_{k \in \mathcal{K}} kv_k \cdot K_1(k, n) = 0\\
            \sum_{k \in \mathcal{K}} kv_k \cdot K_2(k, n) = 0\\
            \ldots\\
            \sum_{k \in \mathcal{K}} kv_k \cdot K_{t-3}(k, n) = 0\\
        \end{cases}
    \end{equation*}
    The base step is done.

    \underline{Induction.} We want to show that if the initial statement is true for $m$, then it is also true for $m+1$. From the starting assumption we have

    $$
    0 = \sum_{k \in \mathcal{K}} k^mv_k \cdot K_{i-m}(k, n) = (i-m)\sum_{k \in \mathcal{K}} k^mv_k \cdot K_{i-m}(k, n) = \sum_{k \in \mathcal{K}} k^mv_k \cdot (i-m)K_{i-m}(k, n).
    $$

    Using \cref{eq:fact1}, we develop this further to

    \begin{equation*}
    \begin{split}
        0 &= \sum_{k \in \mathcal{K}} k^mv_k \cdot \Big( (n - 2k)K_{i - (m+1)}(k, n) - (n - i - m)K_{i - (m + 2)}(k, n) \Big)\\
        & = n \sum_{k \in \mathcal{K}} \Big( k^mv_k \cdot K_{i - (m+1)}(k, n) \Big) - 2\sum_{k \in \mathcal{K}} \Big( k^{m+1}v_k \cdot K_{i - (m+1)}(k, n) \Big)\\& - (n - i - m)\sum_{k \in \mathcal{K}} \Big( k^mv_k \cdot K_{i - (m+2)}(k, n) \Big),
    \end{split}
    \end{equation*}
    for $i = m+2, \ldots t-1$.

    We know that from the step $m-1$, $\sum_{k \in \mathcal{K}} k^mv_k \cdot K_{i - (m+1)}(k, n) = 0$. We also know from the step $m-1$ that for $i = m+3, \ldots, t-1$, $\sum_{k \in \mathcal{K}} k^mv_k \cdot K_{i - (m+2)}(k, n) = 0$. For $i = m + 2$, we end up with $\sum_{k \in \mathcal{K}} k^mv_k \cdot K_0(k, n) = \sum_{k \in \mathcal{K}} k^mv_k$. Since $0 = \sum_{k \in \mathcal{K}} k^{m-1}v_k \cdot K_1(k, n) = \sum_{k \in \mathcal{K}} k^{m-1}v_k \cdot (n - 2k)$, we develop one more time to $n\eqzero{\sum_{k \in \mathcal{K}} k^{m-1}v_k} - 2\sum_{k \in \mathcal{K}}k^mv_k$. Indeed, $\sum_{k \in \mathcal{K}}k^mv_k = 0$ and $\sum_{k \in \mathcal{K}} \Big( k^mv_k \cdot K_{i - (m+2)}(k, n) \Big) = 0$. Thus it follows that
    \begin{equation*}
    \begin{split}
        & = n \eqzero{\sum_{k \in \mathcal{K}} \Big( k^mv_k \cdot K_{i - (m+1)}(k, n) \Big)} - 2\sum_{k \in \mathcal{K}} \Big( k^{m+1}v_k \cdot K_{i - (m+1)}(k, n) \Big)\\& - (n - i - m)\eqzero{\sum_{k \in \mathcal{K}} \Big( k^mv_k \cdot K_{i - (m+2)}(k, n) \Big)}\\
        &= -2\sum_{k \in \mathcal{K}} k^{m+1}v_k \cdot K_{i - (m+1)}(k, n) = \sum_{k \in \mathcal{K}} k^{m+1}v_k \cdot K_{i - (m+1)}(k, n) = 0
    \end{split}
    \end{equation*}
    This concludes the proof that
    $$\sum_{k \in \mathcal{K}} v_k \cdot K_i(k, n) = 0 \implies \sum_{k \in \mathcal{K}} k^mv_k \cdot K_{i-m}(k, n) = 0.$$

        Let us now prove first that
    $$\sum_{k \in \mathcal{K}} k^iv_k = 0 \implies \sum_{k \in \mathcal{K}} k^mv_k \cdot K_{i-m}(k, n) = 0,$$ for $i = 0, \ldots, t-1$ and for any $m = 0, \ldots, i$. We will proceed by induction.

    \underline{Initialization.}

    Let $m = i$, then $\sum_{k \in \mathcal{K}} k^iv_k \cdot K_{i - i}(k, n) = \sum_{k \in \mathcal{K}} k^iv_k \cdot K_{0}(k, n) = \sum_{k \in \mathcal{K}} k^iv_k = 0$.

    Let $m = i - 1$, then
    \begin{equation*}
        \begin{split}
            \sum_{k \in \mathcal{K}} k^{i-1}v_k \cdot K_{i - (i-1)}(k, n) &= \sum_{k \in \mathcal{K}} k^{i-1}v_k \cdot K_{1}(k, n)\\ &= \sum_{k \in \mathcal{K}} k^iv_k \cdot (n - 2k)\\ &= n\sum_{k \in \mathcal{K}} \Big( k^{i-1}v_k \Big) - 2\eqzero{\sum_{k \in \mathcal{K}} \Big( k^iv_k \Big)}.
        \end{split}
    \end{equation*}
    Finally we have $\sum_{k \in \mathcal{K}} k^{i-1}v_k = 0$, for all $i = 1, \ldots, t-1$.

    Let $m = i - 2$, then $\sum_{k \in \mathcal{K}} k^{i-2}v_k \cdot K_{i - (i-2)}(k, n) = \sum_{k \in \mathcal{K}} k^{i-2}v_k \cdot K_{2}(k, n)$. We apply \cref{eq:fact1} to further develop this to
    \begin{equation*}
        \begin{split}
            \sum_{k \in \mathcal{K}} k^{i-2}v_k \cdot K_{2}(k, n) &= \sum_{k \in \mathcal{K}} \Big( k^{i-2}v_k \Big( \frac{1}{2}(n - 2k)K_1(k, n) - \frac{1}{2}(n - 2)K_0(k, n) \Big)\Big)\\
            &= \frac{n}{2}\sum_{k \in \mathcal{K}} \Big( k^{i-2}v_k \cdot K_1(k, n) \Big) - \sum_{k \in \mathcal{K}} \Big( k^{i-1}v_k \cdot K_1(k,n) \Big) - (\frac{n}{2}-1)\sum_{k \in \mathcal{K}} \Big( k^{i-2}v_k \Big).
        \end{split}
    \end{equation*}
    for all $i = 2, \ldots, t-1$.
    From the previous step $m +1 = i - 1$ we had seen that $\sum_{k \in \mathcal{K}} k^{i-1}v_k \cdot K_1(k, n) = 0\quad \forall i\in \{1, \ldots, t-1\}$. As such, using both this and our initial condition we determine that some quantities nullify and we can rewrite the last expression of our development to

    \begin{equation*}
    \begin{split}
        &\sum_{k \in \mathcal{K}} k^{i-2}v_k \cdot K_{2}(k, n)\\
        &= \frac{n}{2}\eqzero{\sum_{k \in \mathcal{K}} \Big( k^{i-2}v_k \cdot K_1(k, n) \Big)} - \eqzero{\sum_{k \in \mathcal{K}} \Big( k^{i-1}v_k \cdot K_1(k,n) \Big)} - (\frac{n}{2}-1)\eqzero{\sum_{k \in \mathcal{K}} \Big( k^{i-2}v_k \Big)} = 0.
    \end{split}
    \end{equation*}

    Let us reformulate \cref{eq:fact1} below
    \begin{equation}\label{eq:fact2}
        2kK_i(k, n) = nK_i(k, n) - (i+1)K_{i+1}(k, n) - (n - i + 1)K_{i-1}(k, n)
    \end{equation}

    \mgnote{I reworked the equality we used to prove the first half of the equivalence in order to expose a $k^{m-1}$ as well as a $K_{i-(m-1)}(k,n)$. I believe the second implication should be proved decreasingly. I'd appreciate feedback.}
    \underline{Induction.} We want to prove that if the proposition is true for $m$, then it is also true for $m -1$. We start from the initial condition and use \cref{eq:fact2} instantly
    \begin{equation*}
        \begin{split}
            0 &= \sum_{k \in \mathcal{K}}k^mv_k \cdot K_{i - m}(v, k)\\
            &= \sum_{k \in \mathcal{K}} \Big( k^mv_k \Big( n\frac{1}{2k} \cdot K_{i - m}(k, n) - \frac{1}{2k}(i - m + 1)K_{i - (m - 1)}(k,n) - \frac{1}{2k}(n- i + m + 1)K_{i-(m+1)}(k,n) \Big)\Big)\\
            &= \frac{n}{2} \eqzero{\sum_{k \in \mathcal{K}}\Big( k^{m-1}v_k \cdot K_{i-m}\Big)} - \frac{i+1}{2}\sum_{k \in \mathcal{K}}\Big( k^{m-1}v_k \cdot K_{i-(m-1)}(k,n)\Big)\\ &- \frac{n - i - m}{2}\sum_{k \in \mathcal{K}} \Big( k^{m-1}v_k \cdot K_{i-(m+1)}(k,n) \Big)\\
        \end{split}
    \end{equation*}

    \mgnote{I think the leftmost quantity is equal to zero from the initial condition. Do you think this method has potential to end in a complete proof?}
\end{proof}


As such, we can reformulate the initial condition as
\begin{equation}\label{eq:general}
    \sum_{k \in \mathcal{K}} k^iv_k \cdot = 0.
\end{equation}

For $i = 0, \ldots t$, we can build the following system using \cref{eq:general} repeatedly.
\begin{equation}
    \begin{cases}
        \sum_{k \in \mathcal{K}} k^0v_k = 0\\
        \sum_{k \in \mathcal{K}} k^1v_k = 0\\
        \sum_{k \in \mathcal{K}} k^2v_k = 0\\
        \ldots\\
        \sum_{k \in \mathcal{K}} k^tv_k = 0.\\
    \end{cases}
\end{equation}

which reformulates our problem as discussing the existence of a vector $(v_{k_1}, \ldots, v_{k_{2^t}})$ such that
$$
(v_{k_1}, \ldots, v_{k_{2^t}})
\begin{bmatrix}
    | & | & | & | & \ldots & |\\
    1 & k & k^2 & k^3 & \ldots & k^t\\
    | & | & | & | & \ldots & |\\
\end{bmatrix} = 0,
$$

using a  $2^t \times (t + 1)$ matrix where each column is the vector of all $k \in \mathcal{K}$.

Let us take an example for $n = 2^i + 2^j$. We have $t = 2$, the matrix would have the following shape
$$
\begin{bmatrix}
    1 & 0 & 0\\
    1 & 2^i & 2^{2i}\\
    1 & 2^j & 2^{2j}\\
    1 & 2^i + 2^j & (2^i + 2^j)^2\\
\end{bmatrix}
$$
that is a Vandermonde matrix. If the conjecture is true, then $\nexists (v_0, v_{2^i}, v_{2^j}, v_{2^i + 2^j}) \in \{-1, 1\}^4$ that belongs to the left kernel of the vandermonde matrix defined by $\mathcal{K}$.
