\begin{proof}
    Similarly as in the proof for Proposition \ref{prop:t=2}, we consider the
    solutions to $\mathbf{W\cdot y = 0}^\intercal_{t+1}$ for $t=2$ and $t=3$.

    Let $n = 2^{i_1} + 2^{i_2} + 2^{i_3}$ with positive integers $i_1 < i_2 < i_3$, this reduces to studying the solutions for
    \begin{equation}\label{eq:vand_t=2}
        \begin{bmatrix}
            1 & 1 & 1 & 1 & 1 & 1 & 1 & 1\\
            0 &  2^{i_1} & 2^{i_2} & (2^{i_1} + 2^{i_2}) & 2^{i_3} & (2^{i_1} + 2^{i_3}) & (2^{i_2} + 2^{i_3}) & (2^{i_1} + 2^{i_2} + 2^{i_3})\\
            0 & 2^{2i_1} & 2^{2i_2} & (2^{i_1} + 2^{i_2})^2 & 2^{2i_3} & (2^{i_1} + 2^{i_3})^2 & (2^{i_2} + 2^{i_3})^2 & (2^{i_1} + 2^{i_2} + 2^{i_3})^2\\
        \end{bmatrix}
        \cdot \mathbf{y} = \begin{bmatrix}
            0\\0\\0
        \end{bmatrix}
    \end{equation}
    and
    \begin{equation}\label{eq:vand_t=3}
        \begin{bmatrix}
            1 & 1 & 1 & 1 & 1 & 1 & 1 & 1\\
            0 &  2^{i_1} & 2^{i_2} & (2^{i_1} + 2^{i_2}) & 2^{i_3} & (2^{i_1} + 2^{i_3}) & (2^{i_2} + 2^{i_3}) & (2^{i_1} + 2^{i_2} + 2^{i_3})\\
            0 & 2^{2i_1} & 2^{2i_2} & (2^{i_1} + 2^{i_2})^2 & 2^{2i_3} & (2^{i_1} + 2^{i_3})^2 & (2^{i_2} + 2^{i_3})^2 & (2^{i_1} + 2^{i_2} + 2^{i_3})^2\\
            0 & 2^{3i_1} & 2^{3i_2} & (2^{i_1} + 2^{i_2})^3 & 2^{3i_3} & (2^{i_1} + 2^{i_3})^3 & (2^{i_2} + 2^{i_3})^3 & (2^{i_1} + 2^{i_2} + 2^{i_3})^3\\
        \end{bmatrix}
        \cdot \mathbf{y} = \begin{bmatrix}
            0\\0\\0\\0
        \end{bmatrix}
    \end{equation}
    where $\mathbf{y} = (y_{000}, y_{001}, y_{010}, y_{011}, y_{100}, y_{101}, y_{110}, y_{111}) \in \{\pm1\}^8$.
    Let us first study the solutions of Equation \ref{eq:vand_t=2}: we will
    iterate through the rows of the matrix to gradually build necessary
    conditions on its solutions.

    Row 1 indicates that $\mathbf{y}$ must be a balanced vector.
    For the condition on row 2, the dot product between $\mathbf{y}$ and the
    second row of the matrix yields the following equation:
    \begin{equation}\label{eq:lin_t=3_row2}
        \begin{split}
            (y_{001} + y_{011} + y_{101} + y_{111}) &\cdot 2^{i_1} +\\
            (y_{010} + y_{011} + y_{110} + y_{111}) &\cdot 2^{i_2} +\\
            (y_{100} + y_{101} + y_{110} + y_{111}) &\cdot 2^{i_3} = 0
        \end{split}
    \end{equation}
    We denote the coefficient sums as $s_1 = y_{001} + y_{011} + y_{101} + y_{111}$,
    $s_2 = y_{010} + y_{011} + y_{110} + y_{111}$ and $s_3 = y_{100} + y_{101} + y_{110} + y_{111}$.
    As such, we can rewrite the above relation as
    \begin{equation*}
        s_1 \cdot 2^{i_1} + s_2 \cdot 2^{i_2} + s_3 \cdot 2^{i_3} = 0
    \end{equation*}
    In fact, $s_1, s_2, s_3$ can only take for value any $\pm 4, \pm 2$ or 0 but
    by applying the constraint of balancedness from row 0, all possible $\binom{8}{4}$
    balanced $\mathbf{y}$ exhibit only five possible multisets of values for
    $s_1, s_2$ and $s_3$ i.e.:
    \begin{itemize}
        \item $\{0, 0, 0\}$
        \item $\{0, 0, \pm2\}$
        \item $\{0, \pm2, \pm2\}$
        \item $\{\pm2, \pm2, \pm2\}$
        \item $\{0, 0, \pm4\}$
    \end{itemize}
    Indeed, any multiset with identical non-zero values can't satisfy Equation
    \ref{eq:lin_t=3_row2} since $i_1 < i_2 < i_3$. Therefore, the condition
    to satisfy the relation on row 1 is that $s_1 = s_2 = s_3 = 0$.
    For the condition on row 3, the dot product between $\mathbf{y}$ and the
    third row of the matrix yields the following equation:
    \begin{equation}\label{eq:lin_t=3_row3}
        \begin{split}
            (y_{001} + y_{011} + y_{101} + y_{111}) &\cdot 2^{i_1} +\\
            (y_{010} + y_{011} + y_{110} + y_{111}) &\cdot 2^{i_2} +\\
            (y_{100} + y_{101} + y_{110} + y_{111}) &\cdot 2^{i_3} +\\
            (y_{011} + y_{111}) &\cdot 2 \cdot 2^{i_1 + i_2} +\\
            (y_{101} + y_{111}) &\cdot 2 \cdot 2^{i_1 + i_3} +\\
            (y_{110} + y_{111}) &\cdot 2 \cdot 2^{i_2 + i_3} = 0.
        \end{split}
    \end{equation}
    Similarly, we denote the last three coefficient sums as $s_{12} = y_{011} + y_{111}$,
    $s_{13} = y_{101} + y_{111}$ and $s_{23} = y_{110} + y_{111}$.
    As such, we can rewrite the above relation as
    \begin{equation*}
        s_1 \cdot 2^{i_1} + s_2 \cdot 2^{i_2} + s_3 \cdot 2^{i_3} + 2s_{12} \cdot 2^{i_1 + i_2} + 2s_{13} \cdot 2^{i_1 + i_3} + 2s_{23} \cdot 2^{i_2 + i_3} = 0
    \end{equation*}
    We determine the value of $s_{12}, s_{13}$ and $s_{23}$. Assume $s_{12} = \pm 2$,
    then $y_{011} = y_{111} = \pm 1$. By considering both the condition of balancedness
    and that $s_1 = s_2 = s_3 = 0$, we necessarily have $y_{101} = y_{110} = -y_{011} = -y_{111}$.
    This doesn't satisfy Equation \ref{eq:lin_t=3_row3}. The same logic applies
    if $y_{101} = y_{111}$ or if $y_{110} = y_{111}$. The only combination that
    works is $y_{011} = y_{101} = y_{110} = -y_{111}$. The solutions that verify
    Equation \ref{eq:lin_t=3_row3} are given by
    $$
    \mathbf{y} = \pm(1, -1, -1, -1, 1, 1, 1, -1).
    $$
    Furthermore, these solutions do not satisfy Equation \ref{eq:vand_t=3}, which implies there is not WAPB function $f$ with $\corr(f) = 3$.
\end{proof}
