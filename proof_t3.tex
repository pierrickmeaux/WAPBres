\begin{proof}
    Similarly as in the proof for Proposition \ref{prop:t=2}, we consider the
    solutions to $\mathbf{W\cdot y = 0}^\intercal_{t+1}$ for $t=2$ and $t=3$.

    Let $n = 2^{i_1} + 2^{i_2} + 2^{i_3}$ with positive integers $i_1 < i_2 < i_3$, this reduces to studying the solutions for
    \begin{equation}\label{eq:vand_t=2}
        \begin{bmatrix}
            1 & 1 & 1 & 1 & 1 & 1 & 1 & 1\\
            0 &  2^{i_1} & 2^{i_2} & (2^{i_1} + 2^{i_2}) & 2^{i_3} & (2^{i_1} + 2^{i_3}) & (2^{i_2} + 2^{i_3}) & (2^{i_1} + 2^{i_2} + 2^{i_3})\\
            0 & 2^{2i_1} & 2^{2i_2} & (2^{i_1} + 2^{i_2})^2 & 2^{2i_3} & (2^{i_1} + 2^{i_3})^2 & (2^{i_2} + 2^{i_3})^2 & (2^{i_1} + 2^{i_2} + 2^{i_3})^2\\
        \end{bmatrix}
        \cdot \mathbf{y} = \begin{bmatrix}
            0\\0\\0
        \end{bmatrix}
    \end{equation}
    and
    \begin{equation}\label{eq:vand_t=3}
        \begin{bmatrix}
            1 & 1 & 1 & 1 & 1 & 1 & 1 & 1\\
            0 &  2^{i_1} & 2^{i_2} & (2^{i_1} + 2^{i_2}) & 2^{i_3} & (2^{i_1} + 2^{i_3}) & (2^{i_2} + 2^{i_3}) & (2^{i_1} + 2^{i_2} + 2^{i_3})\\
            0 & 2^{2i_1} & 2^{2i_2} & (2^{i_1} + 2^{i_2})^2 & 2^{2i_3} & (2^{i_1} + 2^{i_3})^2 & (2^{i_2} + 2^{i_3})^2 & (2^{i_1} + 2^{i_2} + 2^{i_3})^2\\
            0 & 2^{3i_1} & 2^{3i_2} & (2^{i_1} + 2^{i_2})^3 & 2^{3i_3} & (2^{i_1} + 2^{i_3})^3 & (2^{i_2} + 2^{i_3})^3 & (2^{i_1} + 2^{i_2} + 2^{i_3})^3\\
        \end{bmatrix}
        \cdot \mathbf{y} = \begin{bmatrix}
            0\\0\\0\\0
        \end{bmatrix}
    \end{equation}
    where $\mathbf{y} = (y_{000}, y_{001}, y_{010}, y_{011}, y_{100}, y_{101}, y_{110}, y_{111}) \in \{\pm1\}^8$.
   
   \smallskip
   
    Let us first study the solutions of \eqref{eq:vand_t=2}. We will
    iterate through the rows of the matrix to gradually build necessary
    conditions on its solutions. We will refer Row $k$ to as the row indexed by the exponent $k$ for $0\leq k <3$ in $\matW$.
	
    Row $0$ indicates that $\mathbf{y}$ must be a balanced vector, namely, half of its entries equal $1$ and the other half equal $-1$.
    Row $1$ yields the following equation:
    \begin{equation*}
        \begin{split}
            (y_{001} + y_{011} + y_{101} + y_{111}) &\cdot 2^{i_1} +\\
            (y_{010} + y_{011} + y_{110} + y_{111}) &\cdot 2^{i_2} +\\
            (y_{100} + y_{101} + y_{110} + y_{111}) &\cdot 2^{i_3} = 0\\
%            \Longleftrightarrow\\
%            s_1 \cdot 2^{i_1} + s_2 \cdot 2^{i_2} + s_3 &\cdot 2^{i_3} = 0.\\
        \end{split}
    \end{equation*}
    For ease of presentation we denote the sum coefficients of $2^{i_1},2^{i_2},2^{i_3}$ by $s_1, s_2,s_3$, respectively. Then the equation from Row $1$ becomes
    \begin{equation}\label{eq:lin_t=3_row2}
    s_1 \cdot 2^{i_1} + s_2 \cdot 2^{i_2} + s_3 \cdot 2^{i_3} = 0.
    \end{equation}
    It is clear that $s_1, s_2, s_3$ can only take for value any $\pm 4, \pm 2$
    or 0. By further applying the constraint of balancedness from Row $0$, all possible
    $\binom{8}{4}$ balanced $\mathbf{y}$ exhibit only the following possible options for the multiset
     (that allows for same elements) $\{s_1,s_2,s_3\}$:
    \[
    \{0, 0, 0\},  \{0, 0, \pm 2\}, \{0, \pm2, \pm2\}, \{\pm2, \pm2, \pm2\}, \{0, 0, \pm4\}.
    \]
%    \mgnote{We have five and not four multisets of values as I re-verified the results. This doesn't change the proof.}
%    \begin{itemize}
%        \item $\{0, 0, 0\}$
%        \item $\{0, 0, \pm2\}$
%        \item $\{0, \pm2, \pm2\}$
%        \item $\{\pm2, \pm2, \pm2\}$
%        \item $\{0, 0, \pm4\}$
%    \end{itemize}
   This means that each term on the left hand of ~\eqref{eq:lin_t=3_row2} is either zero or a power of $2$. 
	Since $i_1 < i_2 < i_3$, it can be easily verified that the equation~\eqref{eq:lin_t=3_row2} can be true only 
	when $s_1=s_2=s_3=0$. 
%	
%	any multiset with identical nonzero values can't satisfy 
%    Equation \ref{eq:lin_t=3_row2} since . Therefore, the condition
%    to satisfy the relation on row 1 is that $s_1 = s_2 = s_3 = 0$.
    From Row 3 one gets the following equation (after re-arranging the terms):
    \begin{equation}\label{eq:lin_t=3_row3}
        \begin{split}
            (y_{001} + y_{011} + y_{101} + y_{111}) &\cdot 2^{i_1} +\\
            (y_{010} + y_{011} + y_{110} + y_{111}) &\cdot 2^{i_2} +\\
            (y_{100} + y_{101} + y_{110} + y_{111}) &\cdot 2^{i_3} +\\
            (y_{011} + y_{111}) &\cdot 2 \cdot 2^{i_1 + i_2} +\\
            (y_{101} + y_{111}) &\cdot 2 \cdot 2^{i_1 + i_3} +\\
            (y_{110} + y_{111}) &\cdot 2 \cdot 2^{i_2 + i_3} = 0.\\
%            \Longleftrightarrow\\
%            s_1 \cdot 2^{i_1} + s_2 \cdot 2^{i_2} + s_3 \cdot 2^{i_3} + 2s_{12} \cdot 2^{i_1 + i_2} + 2s_{13} &\cdot 2^{i_1 + i_3} + 2s_{23} \cdot 2^{i_2 + i_3} = 0.\\
        \end{split}
    \end{equation}
    After applying the preceding constraints, namely, balancedness of $\mathbf{y}$ from Row 0 and $s_1=s_2=s_3=0$ from Row 1, we obtain 
    \[
     2s_{12} \cdot 2^{i_1 + i_2} + 2s_{13} \cdot 2^{i_1 + i_3} + 2s_{23} \cdot 2^{i_2 + i_3} = 0, 
    \] where $s_{12}, s_{13},s_{23}$ correspond to the sum coefficient of $2^{i_{1}+i_2}, 2^{i_{1}+i_3}$ and $2^{i_{1}+i_3}$, respectively. 
    We now discuss the values of $s_{12}, s_{13}$ and $s_{23}$. Assume $s_{12} = \pm 2$.
    Then $y_{011} = y_{111} = \pm 1$. By considering both the condition of balancedness
    and that $s_1 = s_2 = s_3 = 0$, we necessarily have $y_{101} = y_{110} = -y_{011} = -y_{111}$.
    This doesn't satisfy Eq.~\ref{eq:lin_t=3_row3}. The same logic applies
    if $y_{101} = y_{111}$ or if $y_{110} = y_{111}$. The only combination that
    works is $y_{011} = y_{101} = y_{110} = -y_{111}$, thereby giving the following solutions to Eq.~\eqref{eq:lin_t=3_row3} 
    $$
    \mathbf{y} = \pm(1, -1, -1, -1, 1, 1, 1, -1).
    $$
    Furthermore, these solutions do not satisfy the last equation in \eqref{eq:vand_t=3}. That is to say, there is not WAPB function $f$ with $\corr(f) = 3$.
\end{proof}
