\begin{proof}
We consider solutions to the linear systems $\matW \cdot \vecy = \mathbf{0}^\intercal_t$
and $\matW \cdot \vecy = \mathbf{0}^\intercal_{t+1}$ where
$\vecy = (y_{0\ldots00}, y_{0\ldots01}, y_{0\ldots10}, \ldots, y_{1\ldots10}, y_{1\ldots11}) \in \{\pm1\}^{2^t}$.
\begin{equation}\label{eq:vand_t}
        \begin{bmatrix}
            1 & 1 & 1 & 1 & \ldots & 1\\
            0 &  2^{i_1} & 2^{i_2} & (2^{i_1} + 2^{i_2}) & \ldots & (2^{i_1} + \ldots + 2^{i_t})\\
            \vdots & \vdots & \vdots & \vdots &  & \vdots\\
            0 & 2^{(t-1)i_1} & 2^{(t-1)i_2} & (2^{i_1} + 2^{i_2})^{t-1} & \ldots & (2^{i_1} + \ldots + 2^{i_t})^{t-1}\\
        \end{bmatrix}
        \cdot \mathbf{y} = \begin{bmatrix}
            0\\0\\\vdots\\0
        \end{bmatrix}.
    \end{equation}
For each $y_i$ we can write $i = \sum_{\delta=0}^{t-1}i_\delta 2^\delta$.
Let us introduce the notation $\vecy_B = \{y_i \mid i_\delta = 1, \forall \delta \in B\}$
for some $B \subseteq \{0, 1, \ldots, t-1\}$.

We will proceed by induction on Rows $1$ to $t-1$ of $\matW$ where Row $r$ is
the row which all coefficients of $\matW$ are exponentiated by $r$. We want to
prove that all sum coefficient must be null for each linear relation induced by
the first $r$ rows. Row $0$ forces $\vecy$ to be a balanced vector with
coefficients in $\{\pm 1\}$.

\underline{Initialization.}

Let $r = 1$. Row $1$ induces the equation below
\begin{equation*}
    \sum_{y_i \in \vecy_{\{1\}}} (y_i) \cdot 2^{i_1} + \sum_{y_i \in \vecy_{\{2\}}} (y_i) \cdot 2^{i_2} + \ldots + \sum_{y_i \in \vecy_{\{t\}}} (y_i) \cdot 2^{i_t} = 0.
\end{equation*}
For ease of presentation we denote the sum coefficients of $2^{i_j}$ by $s_j$.
Then the equation from Row $1$ becomes
\begin{equation}\label{eq:lin_t_row1}
    s_1 \cdot 2^{i_1} + s_2 \cdot 2^{i_2} + \ldots + s_t \cdot 2^{i_t} = 0.
\end{equation}
Because $\vecy$ is balanced, it is clear that any $s_j$ can take for values any even number or its opposite
in the range $[0, 2^{t-1}]$ because $|\vecy_B| = 2^{t-|B|}$. Since $\lambda_n \geq t$,
there doesn't exist any $s_j \neq 0$ to satisfy Equation~\ref{eq:lin_t_row1}.

Let $r = 2$. From Row 2 one gets the following equation after re-arranging the
terms
\begin{equation*}
    \begin{split}
        &\sum_{y_i \in \vecy_{\{1\}}} (y_i) \cdot 2^{i_1} + \sum_{y_i \in \vecy_{\{2\}}} (y_i) \cdot 2^{i_2} + \ldots + \sum_{y_i \in \vecy_{\{t\}}} (y_i) \cdot 2^{i_t} +\\
        &2\sum_{y_i \in \vecy_{\{1,2\}}} (y_i) \cdot 2^{i_1 + i_2} + 2\sum_{y_i \in \vecy_{\{1,3\}}} (y_i) \cdot 2^{i_1 + i_3} + \ldots + 2\sum_{y_i \in \vecy_{\{t-1, t\}}} (y_i) \cdot 2^{i_{t-1} + i_{t}} = 0.
    \end{split}
\end{equation*}
We use the previously defined notations and knowledge that $s_1 = s_2 = \ldots = s_t = 0$
to simplify it into
\begin{equation}\label{eq:lin_t_row2}
    2s_{12} \cdot 2^{i_1 + i_2} + 2s_{13} \cdot 2^{i_1 + i_3} + \ldots + 2s_{(t-1)(t)} \cdot 2^{i_{t-2} + i_{t-1}} = 0.
\end{equation}
Because $\vecy$ is balanced, it is clear that any $s_j$ can take for values any even number or its opposite
in the range $[0, 2^{t-2}]$ because $|\vecy_B| = 2^{t-|B|}$. Since $i_j + i_k - i_l - i_m \geq t$
for all $j < k < l < m \leq t$, there doesn't exist any $s_j \neq 0$ to satisfy
Equation~\ref{eq:lin_t_row2}. The base step is done.

\underline{Induction:} we shall prove that if the statement is true for $r-1$,
then it is true for $r$. From Row $r$ and using Lemma~\ref{lem:Represention_Equations} one gets the following equation
\begin{equation*}
    \sum_{\ell \preceq n} \ell^r y_\ell = \sum_{m=1}^t\sum\limits_{\substack{\{j_1,\dots, j_m\}\subseteq[t]\\ r_{j_1}+\cdots+ r_{j_m}=r\\ r_{j_1}, \dots, r_{j_m} \geq 1}} \frac{r!}{r_{j_1}!\cdots r_{j_m}!} \left(\sum_{\substack{\{i_{j_1},\dots, i_{j_m}\} \subseteq supp(\ell)\\ \ell\preceq n}} y_{\ell} \right) x_{j_1}^{r_{j_1}}\dots x_{j_m}^{r_{j_m}} = 0.
\end{equation*}

\begin{equation*}
    s_1
\end{equation*}
We know from Row $r-1$ that $s_1, \ldots, s_t, s_{12}, \ldots, s_{(t-1)(t)}, \ldots, s_{(t-(r-1))\ldots(t-1)(t)} = 0$.
We then reformulate the above equation into
\begin{equation}\label{eq:lin_t_rowr}
    s_{12\ldots r} \cdot 2^{i_1 + i_2 + \ldots + i_r} + \ldots + s_{(t-r)(t-r+1)\ldots(t)} \cdot 2^{i_{t-r} + i_{t-r+1} + \ldots + i_{t}} = 0.
\end{equation}


\end{proof}
