\begin{proof}
From Theorem~\ref{thm:t_corrector_Vandemonde_matrix} we study the correction
order of such functions by considering solutions to the linear systems
$\matW \cdot \vecy = \mathbf{0}^\intercal_t$ and
$\matW \cdot \vecy = \mathbf{0}^\intercal_{t+1}$ where
$\vecy = (y_{0\ldots00}, y_{0\ldots01}, y_{0\ldots10}, \ldots, y_{1\ldots10}, y_{1\ldots11}) \in \{\pm1\}^{2^t}$.
\begin{equation}\label{eq:vand_t}
        \begin{bmatrix}
            1 & 1 & 1 & 1 & \ldots & 1\\
            0 &  2^{i_1} & 2^{i_2} & (2^{i_1} + 2^{i_2}) & \ldots & (2^{i_1} + \ldots + 2^{i_t})\\
            \vdots & \vdots & \vdots & \vdots &  & \vdots\\
            0 & 2^{(t-1)i_1} & 2^{(t-1)i_2} & (2^{i_1} + 2^{i_2})^{t-1} & \ldots & (2^{i_1} + \ldots + 2^{i_t})^{t-1}\\
        \end{bmatrix}
        \cdot \mathbf{y} = \begin{bmatrix}
            0\\0\\\vdots\\0
        \end{bmatrix}.
    \end{equation}
For each $y_i$ we can write $i = \sum_{\delta=0}^{t-1}i_\delta 2^\delta$.
Let us introduce the notation $\vecy_\Delta = \{y_i \mid i_\delta = 1, \forall \delta \in \Delta\}$
for some $\Delta \subseteq \{0, 1, \ldots, t-1\}$.

We will proceed by induction on Rows $1$ to $t-1$ of $\matW$ where Row $r$ is
the row in which all coefficients of $\matW$ are exponentiated by $r$. We want
to prove that all sum coefficients must be null for each linear relation induced
by the first $r$ rows. Row $0$ forces $\vecy$ to be a balanced vector with
coefficients in $\{\pm 1\}$.

\underline{Initialization.}

Let $r = 1$. Row $1$ induces the equation below
\begin{equation*}
    \left(\sum_{y_i \in \vecy_{\{1\}}} y_i\right) \cdot 2^{i_1} + \left(\sum_{y_i \in \vecy_{\{2\}}} y_i\right) \cdot 2^{i_2} + \cdots + \left(\sum_{y_i \in \vecy_{\{t\}}} y_i\right) \cdot 2^{i_t} = 0.
\end{equation*}
For ease of presentation we denote the sum coefficients of $2^{i_j}$ by $s_j$.
Then the equation from Row $1$ becomes
\begin{equation}\label{eq:lin_t_row1}
    s_1 \cdot 2^{i_1} + s_2 \cdot 2^{i_2} + \cdots + s_t \cdot 2^{i_t} = 0.
\end{equation}
Because $\vecy$ is balanced, it is clear that any $s_j$ can take for values any
even number or its opposite in the range $[0, 2^{t-1}]$ because
$|\vecy_\Delta| = 2^{t-|\Delta|}$. From the initial condition,
$[i_1, \ldots, i_t]$ is a $t$-gap $B_{t-1}$ sequence and since $s_i < 2^{t}$ and
because the gap between two any two $i_j$ is at least $t$, there doesn't exist
any $s_i \neq 0$ satisfying Equation~\ref{eq:lin_t_row1}.

Let $r = 2$. From Row 2 one gets the following equation after re-arranging the
terms
\begin{equation*}
    \begin{split}
        &\left(\sum_{y_i \in \vecy_{\{1\}}} y_i\right) \cdot 2^{i_1} + \left(\sum_{y_i \in \vecy_{\{2\}}} y_i\right) \cdot 2^{i_2} + \cdots + \left(\sum_{y_i \in \vecy_{\{t\}}} y_i\right) \cdot 2^{i_t} +\\
        &2\left(\sum_{y_i \in \vecy_{\{1,2\}}} y_i\right) \cdot 2^{i_1 + i_2} + 2\left(\sum_{y_i \in \vecy_{\{1,3\}}} y_i\right) \cdot 2^{i_1 + i_3} + \cdots + 2\left(\sum_{y_i \in \vecy_{\{t-1, t\}}} y_i\right) \cdot 2^{i_{t-1} + i_{t}} = 0.
    \end{split}
\end{equation*}
We use the previously defined notations and knowledge that
$s_1 = s_2 = \cdots = s_t = 0$ to simplify it into
\begin{equation}\label{eq:lin_t_row2}
    2s_{12} \cdot 2^{i_1 + i_2} + 2s_{13} \cdot 2^{i_1 + i_3} + \cdots + 2s_{(t-1)(t)} \cdot 2^{i_{t-1} + i_{t}} = 0.
\end{equation}
Similarly as above, we see that any $s_j$ can take for values any even number or
its opposite in the range $[0, 2^{t-2}]$ and we make use of the initial
condition that $[i_1, \ldots, i_t]$ is a $t$-gap $B_{t-1}$ sequence and bring
the same argument to prove there doesn't exist any $s_i \neq 0$ satisfying
Equation~\ref{eq:lin_t_row2}.

\underline{Induction:} we shall prove that if the statement is true for $r-1$,
then it is true for $r$. From Row $r$ and using Lemma~\ref{lem:Represention_Equations} one gets the following equation
\begin{equation*}
    \sum_{\ell \preceq n} \ell^r y_\ell = \sum_{m=1}^t\sum\limits_{\substack{\{j_1,\dots, j_m\}\subseteq[t]\\ r_{j_1}+\cdots+ r_{j_m}=r\\ r_{j_1}, \dots, r_{j_m} \geq 1}} \frac{r!}{r_{j_1}!\cdots r_{j_m}!} \left(\sum_{\substack{\{i_{j_1},\dots, i_{j_m}\} \subseteq supp(\ell)\\ \ell\preceq n}} y_{\ell} \right) x_{j_1}^{r_{j_1}}\dots x_{j_m}^{r_{j_m}} = 0.
\end{equation*}

Let us denote the multinomial coefficient by $\mu_{1\ldots m} = \frac{r!}{r_{j_1}!\cdots r_{j_m}!}$
and the previously defined notations $s_{1\ldots m} = \sum (y_\ell)$. We rewrite
the above equation into its rearranged form

\begin{equation*}
    \begin{split}
        \mu_1 s_1 2^{i_1} + \cdots + \mu_1 s_t 2^t &+\\
        \mu_{12} s_{12} 2^{i_1 + i_2} + \cdots + \mu_{12} s_{(t-1)(t)}+ 2^{i_{t-1} + i_{t}} &+\\
        \cdots &+\\
        \mu_{12\ldots r} s_{12\ldots r} 2^{i_1 + i_2 + \cdots + i_r} + \cdots + \mu_{12\ldots r} s_{(t-r)(t-r+1)\cdots(t)} \cdot 2^{i_{t-r} + i_{t-r+1} + \cdots + i_{t}} &= 0.
    \end{split}
\end{equation*}

We know from Row $r-1$ that $s_1, \ldots, s_t, s_{12}, \ldots, s_{(t-1)(t)}, \ldots, s_{(t-(r-1))\ldots(t-1)(t)} = 0$.
We then reformulate the above equation into
\begin{equation}\label{eq:lin_t_rowr}
    \mu_{12\ldots r} s_{12\ldots r} 2^{i_1 + i_2 + \cdots + i_r} + \cdots + \mu_{12\ldots r} s_{(t-r)(t-r+1)\cdots(t)} \cdot 2^{i_{t-r} + i_{t-r+1} + \cdots + i_{t}} = 0.
\end{equation}
In a similar fashion, we see that any $s_j$ can take for values any even number
or its opposite in the range $[0, 2^{t-r}]$ and we make use of the initial
condition that $[i_1, \ldots, i_t]$ is a $t$-gap $B_{t-1}$ sequence and bring
the same argument to prove there doesn't exist any $s_i \neq 0$ satisfying
Equation~\ref{eq:lin_t_rowr}. This concludes the proof.
\end{proof}
