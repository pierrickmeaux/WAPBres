\begin{proof}
We consider solutions to the linear systems $\matW \cdot \vecy = \mathbf{0}^\intercal_t$
and $\matW \cdot \vecy = \mathbf{0}^\intercal_{t+1}$ where
$\vecy = (y_{0\ldots00}, y_{0\ldots01}, y_{0\ldots10}, \ldots, y_{1\ldots10}, y_{1\ldots11}) \in \{\pm1\}^{2^t}$.
\begin{equation}\label{eq:vand_t}
        \begin{bmatrix}
            1 & 1 & 1 & 1 & \ldots & 1\\
            0 &  2^{i_1} & 2^{i_2} & (2^{i_1} + 2^{i_2}) & \ldots & (2^{i_1} + \ldots + 2^{i_t})\\
            \vdots & \vdots & \vdots & \vdots &  & \vdots\\
            0 & 2^{(t-1)i_1} & 2^{(t-1)i_2} & (2^{i_1} + 2^{i_2})^{t-1} & \ldots & (2^{i_1} + \ldots + 2^{i_t})^{t-1}\\
        \end{bmatrix}
        \cdot \mathbf{y} = \begin{bmatrix}
            0\\0\\\vdots\\0
        \end{bmatrix}
    \end{equation}
For each $y_i$ we can write $i = \sum_{\delta=0}^{t-1}i_\delta 2^\delta$.
Let us introduce the notation $\vecy_B = \{y_i \mid i_\delta = 1, \forall \delta \in B\}$
for some $B \subseteq \{0, 1, \ldots, t-1\}$.
We will now proceed by induction on the rows of of $\matW$. We want to prove the
linear sum is null at each exponentiation and that it depends on the previous step.

Row $0$ forces $\vecy$ to be a balanced vectors with coefficients in $\{\pm 1\}$.
Row $1$ induces the equation below
\begin{equation*}
    \sum_{y_i \in \vecy_{\{1\}}} (y_i) \cdot 2^{i_1} + \sum_{y_i \in \vecy_{\{2\}}} (y_i) \cdot 2^{i_2} + \ldots + \sum_{y_i \in \vecy_{\{t\}}} (y_i) \cdot 2^{i_t} = 0.
\end{equation*}

For ease of presentation we denote the sum coefficients of $2^{i_j}$ by $s_j$.
Then the equation from Row $1$ becomes
\begin{equation}\label{eq:lin_t_row1}
    s_1 \cdot 2^{i_1} + s_2 \cdot 2^{i_2} + \ldots + s_t \cdot 2^{i_t} = 0.
\end{equation}

It is clear that any $s_j$ can take for values any even number or its opposite
in the range $[0, 2^{t-1}]$. Since any $i_j - i_{j+1} \geq t$, there doesn't
exist any $s_j \neq 0$ to satisfy Equation~\ref{eq:lin_t_row1}.

From Row 2 one gets the following equation after re-arranging the terms
\begin{equation*}
    \begin{split}
        &\sum_{y_i \in \vecy_{\{1\}}} (y_i) \cdot 2^{i_1} + \sum_{y_i \in \vecy_{\{2\}}} (y_i) \cdot 2^{i_2} + \ldots + \sum_{y_i \in \vecy_{\{t\}}} (y_i) \cdot 2^{i_t} +\\
        &\sum_{y_i \in \vecy_{\{1,2\}}} (y_i) \cdot 2^{i_1 + i_2} + \sum_{y_i \in \vecy_{\{1,3\}}} (y_i) \cdot 2^{i_1 + i_3} + \ldots + \sum_{y_i \in \vecy_{\{t-2, t-1\}}} (y_i) \cdot 2^{i_{t-2} + i_{t-1}} = 0.
    \end{split}
\end{equation*}

After replacing the notations and substituting null quantities and we obtain
\begin{equation}\label{eq:lin_t_row2}
    s_{12} \cdot 2^{i_1 + i_2} + s_{13} \cdot 2^{i_1 + i_3} + \ldots + s_{(t-2)(t-1)} \cdot 2^{i_{t-2} + i_{t-1}} = 0.
\end{equation}

It is clear that any $s_j$ can take for values any even number or its opposite
in the range $[0, 2^{t-2}]$. Since any $i_j + i_k - i_l - i_m \geq t$, there doesn't
exist any $s_j \neq 0$ to satisfy Equation~\ref{eq:lin_t_row2}.

\end{proof}
